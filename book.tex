\documentclass[openany,ngerman]{book}

\usepackage[utf8]{inputenc}%(only for the pdftex engine)
%\RequirePackage[no-math]{fontspec}[2017/03/31]%(only for the luatex or the xetex engine)
\usepackage[small]{dgruyter}
\usepackage{microtype}


\author{Max Mustermann}
\title{Funktion als Teil der literarischen Darstellung}
\distributionseries{De\,Gruyter Studium}
\seriestitle{Übergreifende Fragestellung zu Methode und Darstellungsweise}
\seriessubtitle{Studien und Vorunter suchungen zur kommentierten Überlieferung und Wirkung der Abweichungen vom Text~der Ausgabe und Fassung}
\serieseditor{Max Mustermann und Moritz Maier}
\seriesvolume{5/31}
\subtitle{Unterscheidung als Hilfs- und Orientierungsbegriff zur Dekodierung}
\editor{Max Mustermann als Editor}
\collaborator{Mustermann Stiftung}
\edition{2. Auf\/lage}
%\publisherlogo{dg-mouton}
\classification[Mathematics Subject Classification 2010]{35-02, 65-02, 65C30, 65C05, 65N35, 65N75, 65N80}
\authorinfo{Auch beim Konzept der ästhetischen Bildung von Wissen und dessen möglichst rasche und erfolgsorientierte Anwendung verspielen Einsichten und Gewinne ohne den Bezug auf die um 1900 entwickelten Argumentationen.\\\vskip\baselineskip Auch umfasst die Untersuchung in der Hauptsache den Zeitraum zwischen dem Inkrafttreten und der Darstellung in seiner heute geltenden Fassung. Ihre Funktion als Teil der literarischen Darstellung und narrativen Technik enthält auch einen vollständigen textkritischen Apparat und ein Verzeichnis der Stellen.}
\copyrighttext{Copyright-Text}
\isbn{978-3-11-021808-4}
\eisbnpdf{978-3-11-021809-1}
\issn{0179-0986}
\cover{Cover-Firma}
\typesetter{le-tex publishing services GmbH, Leipzig}
\printbind{Druckerei XYZ}

%\includeonly{%
%preface,
%chapter01,
%...
%}

\begin{document}
\frontmatter
\maketitle
\dedication{...\\ ...}
%\include{preface}


\mainmatter

%\include{chapter01}

\backmatter
%\bibliographystyle{plain}
%\printbibliography[env=bibnumeric]
\end{document}


%here comes an example for a contribution header:
\contribution

  \contributionauthor{Max Müller}
  \affil{Institute of Marine Biology, National Taiwan Ocean University, 2 Pei-Ning Road Keelung
    20224, Taiwan (R.O.C), e-mail: klpang@ntou.edu.tw}
  \runningauthor{Müller}
  \contributiontitle{Was ist so anders am Neuroenhancement? Pharmakologische und mentale Selbstveränderung im ethischen Vergleich}
  \runningtitle{Was ist so anders am Neuroenhancement?}
  \contributionsubtitle{Pharmakologische und mentale Selbstveränderung im ethischen Vergleich}
  \abstract{Konzept der ästhetischen Bildung von Wissen und dessen möglichst rasche und
    erfolgsorientierte Anwendung verspielen Einsichten und Gewinne ohne den Bezug auf die um 1900
    entwickelten Argumentationen. Auch umfasst der literarischen Übergreifende Fragestellung zur
    besseren Darstellungsweise enthält auch einen vollständigen textkritischen Apparat und ein
    Verzeichnis der Stellen.}
  \keywords{Funktion, Überlieferung, Untersuchung, Hauptsache, Zeitraum, Darstellung, Inkrafttreten,
    Fassung}
  \classification{65C05, 62M20, 93E11, 62F15, 86A22}

\makecontributiontitle
  \DOI{10.1515/futur-2012-0001}
